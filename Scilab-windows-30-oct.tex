\documentclass[11pt]{article}
\title{Instructions to use Spoken Tutorials \\
Scilab for Linux Users}
\author{Anuradha Amrutkar}
\date{23 August 2011}
\topmargin -1.2 in
\textheight 10.5 in
\textwidth 7in
\oddsidemargin -0.25in
\evensidemargin 0in
\usepackage{graphicx,multicol}
\newenvironment{enumcpt}{\begin{enumerate} \topsep 0pt \partopsep 0pt 
                        \parsep 0pt
                        \itemsep 0pt \leftmargin -1in \rightmargin 0pt
                        }{\end{enumerate}}

\pagestyle{empty}
\thispagestyle{empty}
\begin{document}
\begin{minipage}[t]{0.15\textwidth}
\includegraphics[width=\linewidth]{3t-logo.pdf}
\end{minipage} \hfill
\begin{minipage}[t]{0.68\textwidth}
\begin{center}
\vspace{-0.7in}
\Large
Spoken Tutorial on Scilab for Windows Users \\
\large
Spoken Tutorial Team \\
IIT Bombay \\
%23 August 2011
\end{center}
\end{minipage} \hfill
\begin{minipage}[t]{0.12\textwidth}
\includegraphics[width=\linewidth]{st-logo.jpg}
\end{minipage}

\begin{multicols}{2}

\section*{The procedure to practise}
\begin{enumcpt}
\item You are given a set of spoken tutorials and files, available in
  the directory {\tt Scilab\_Workshop}, on your {\tt Desktop}.  You
  may also have been given these in a CD.
\item Follow the tutorials in the sequence given below.
\end{enumcpt}

Please find out from the workshop co-ordinator whether Scilab is
already installed.  If so, please skip the next section and go to
Section 2.

\section{Scilab Installation (Windows):}
Tutorial required: {\tt 01-Installation-Windows.wmv}
\\ In this tutorial you will learn how to install Scilab on Windows Operating System.

\begin{enumcpt}
\item Locate the folder {\tt Scilab\_Workshop} that is available on
  your Desktop.  Go inside the sub-folder {\tt
    01 Installation} and right click on {\tt
    01-Installation-Windows.wmv}, select {\tt open with} option,
  choose the {\tt VLC Media Player}, and listen to the tutorial.
\item Follow the tutorial as shown in the video and install
  Scilab. 
\end{enumcpt}

\section{Getting Started with Scilab:}
Tutorial required: {\tt 02-Getting-Started.wmv} \\ In this tutorial you will learn about some of the very basic functionalities of Scilab.

\begin{enumcpt}
\item Double Click on the Scilab Shortcut icon on your Desktop to
  launch Scilab.  This will open the Scilab console window on your
  computer. 
\item Locate the folder {\tt Scilab\_Workshop} that is available on
  your Desktop.  
  \begin{enumcpt}
  \item Go inside the sub-folder \\ {\tt 02 Getting Started} and right
    click on {\tt 02-Getting-Started.wmv}.
  \item Choose {\tt Open with} and select {\tt VLC
    Media Player} to play the tutorial.
\end{enumcpt}
\item Follow the tutorial and reproduce all the commands on your {\tt
    Scilab Console} as shown in the video.  Use pause, rewind and
  play, as required.
% \item At 4:17, pause the video and work out the indicated assignment.
%   Solve as many assignment problems, as time permits.
\item At 5:34, change the directory to {\tt Desktop} before giving the
  {\tt diary()} command.  You may also do this from
  any other directory location, where you have the write access.
\item At 5:34, the tutorial explains the {\tt diary()} command.
  Please note that only the SUBSEQUENT commands will be saved in the
  file.  All commands given BEFORE the {\tt diary()} command will NOT be
  saved. 
\item At 6:00, pause the video and work out the assignment given in
  the video or in the assignment sheet or in both.
  Solve as many assignment problems, as time permits.
\item At 7:49, {\tt D-17} denotes $10^{-17}$.
\item After completing this tutorial, please go to the next tutorial,
  namely, {\tt Vector Operations}.

\end{enumcpt}

\section{Vector Operations:}
Tutorial required: {\tt 03-Vector-Operations.wmv}
\\
This tutorial explains how to define vectors and matrices in Scilab,
and how to perform some basic arithmetic operations on vectors. 
 
\begin{enumcpt}
\item Open Scilab as mentioned in Section~2.
\item As explained in Section~2, point 2, open {\tt
    03-Vector-Operations.wmv}, and play it.
\item At 2:49 pause the video and work out the assignment given in
  it or in the assignment sheet or in both.
  Solve as many assignment problems, as time permits.
\item Please follow the tutorial and reproduce all the commands as
  shown in the video. 
\end{enumcpt}

\section{Matrix Operations:}
Tutorial required: {\tt 04-Matrix-Operations.wmv } \\ In this tutorial
you will learn some of the most basic but frequently used arithmetic
operations on matrices. The main motivation of this tutorial is to
give you a head start about using vectors and matrices in Scilab. 

\begin{enumcpt}
\item Open Scilab as mentioned in Section-2, point-1
\item Open {\tt 04-Matrix-Operations.wmv } as explained in Section-2,
  point-2.
\item At 1:55 - Notice the type of bracket used in the command, it is
  round bracket and not the square bracket. 
\item At 6:10, Solve the assignments as suggested in the tutorial.  Solve as
  many as possible.
\item At 11:18 - Matrix A represents a matrix of co-efficients of x1, x2, x3 in the equations.
\item At 11:46 - Matrix b represnts a matrix of constants in the equations.


\end{enumcpt}

\section{Scripts and Functions:}
Tutorial required:{\tt 05-Scripts-and-Functions.wmv }
\\ In this tutorial you will learn how to write script files (.sci files) and function files (.sce files). Also we will see how to load and execute user-defined functions in the Scilab console.
\begin{enumcpt}
\item Open Scilab and also play \\
{\tt 05-Scripts-and-Functions.wmv} using the procedure suggested in the
earlier tutorials.
\item At 2:10, the video talks about a {\tt Scilab editor}.  In Scilab
  versions 5.3 and above, this editor is called {\tt SciNotes}. % Open
  %the file {\tt helloworld.sce}, available inside the sub-folder
  %Scripts-and-Functions. You may also 
  Pause the video here type the commands shown in the {\tt helloworld.sce} file in the editor yourself and save the file as {\tt helloworld.sce} on the {\tt Desktop}.  
%This will save this file on your {\tt Desktop}. 
\item At 3:25 - Type {\tt pwd } to check the present working directory; change the directory using {\tt Change Directory} shortcut icon to the {\tt Desktop} or the directory where you have saved the {\tt helloworld.sce } file before using the {\tt exec } command.
\item At 4:00, in the video a file named {\tt function.sci} is opened. 
	\begin{enumcpt}
	\item At 4:03, pause the video, type the function statements shown in 			the file yourself in the editior and save it on your {\tt Desktop}.
	\end{enumcpt}
\item At 5:51, in the video a file named {\tt 2outputs.sci} is opened, pause the video, type the function statements shown in the file yourself in the editior and save it on your {\tt Desktop}.
\item At 9:47, in the video a file named {\tt inline.sci} is opened, pause the video, type the function statements shown in the file yourself in the editior and save it on your {\tt Desktop}.
\item Solve the assignments as given in the Assignment Sheet. Solve as many as possible.
\end{enumcpt}

\section{Conditional Branching:}
Tutorial required: \\ {\tt 06-Conditional-Branching-mov.mov} \\In this tutorial we will discuss two types of Conditional constructs in Scilab, which are, the ``if-then-else" construct and the ``select-case" conditional construct. 

\begin{enumcpt}
\item Open Scilab as mentioned in Section-2, point-1
\item Open Scilab and also play \\
{\tt 06-Conditional-Branching-mov.mov} using the procedure suggested in the
earlier tutorials.
\item Please follow the tutorial and reproduce all the commands as
  shown in the video.
\item Solve the assignments as given in the Assignment Sheet.  Solve as many as possible.
\end{enumcpt}

\section{Iterations:}
Tutorial required: {\tt 07-Iterations-mov.mov} \\ 
In this tutorial we will discuss about the for-loop and the while-loop, used for iterative calulations.
\begin{enumcpt}
\item Open Scilab as mentioned in Section-2, point-1
\item Open {\tt 07-Iterations-mov.mov} using the procedure suggested in the
earlier tutorials.
\item Please follow the tutorial and reproduce all the commands as
  shown in the video.
\item Solve the assignments as given in the Assignment Sheet.  Solve as many as possible.
\end{enumcpt}

\section{Plotting 2D Graphs:}
Tutorial required: {\tt 08-Plotting2dGraphs-mov.mov} 
\\In this tutorial we will learn about the in-built functions like linspace(), plot(), plot2d(). We will also demonstrate how to change the properties of figures using various commands.
\begin{enumcpt}
\item Open Scilab as mentioned in Section-2, point-1
\item Open {\tt 08-Plotting2dGraphs-mov.mov} as explained in Section-2, point-3
\item Please follow the tutorial and reproduce all the commands as
  shown in the video.
\item Solve the assignments as given in the Assignment Sheet.  Solve as many as possible.
\end{enumcpt}

\section{Ordinary Differential Equations:}
Tutorial required: {\tt 09-odes.mov} \\ 
This tutorial is designed to familiarize the participants with solving ordinary differential equations using the Scilab procedure, ``ode".
\begin{enumcpt}
\item Open Scilab as mentioned in Section-2, point-1
\item Open {\tt 09-odes.mov} as explained in Section-2, point-3
\item Please follow the tutorial and reproduce all the commands as
  shown in the video.
\item Solve the assignments as given in the Assignment Sheet.  Solve as many as possible.
\end{enumcpt}

\section{Polynomials:}
Tutorial required: {\tt 10-polynomials.mov} \\ 
In this tutorial we will see the use of Scilab to create polynomials, find
their roots and perform operations on them such as addition, subtraction, multiplication, division, simplification, etc.
\begin{enumcpt}
\item Open Scilab as mentioned in Section-2, point-1
\item Open {\tt 10-polynomials.mov} as explained in Section-2, point-3
\item Please follow the tutorial and reproduce all the commands as
  shown in the video.
\item Solve the assignments as given in the Assignment Sheet.  Solve as many as possible.
\end{enumcpt}


\section{Why Scilab:}
Tutorial required: {\tt 00-Why-Scilab.wmv} \\
In this tutorial you will get an overview about the capabilities of Scilab and the benefits of using Scilab.





\end{multicols}


\end{document}
\section{The software that you will need are:}
\pagestyle{empty}
\thispagestyle{empty}
